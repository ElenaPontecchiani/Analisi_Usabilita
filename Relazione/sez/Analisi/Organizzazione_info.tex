
Oltre al pessimo impatto visuale anche i contenuti delle sezioni sono organizzate in maniera confusionaria: non vi è una buona separazione dei contenuti. Per esempio la voce "occupazione aule e laboratori" compare nella sezione "Didattica" e non all'interno di "strutture e servizi", come logicamente dovrebbe essere.\\
Inoltre le varie sezioni non sono mutamente esclusive (vedi "In primo piano" e "I nostri eventi", non cambia nulla), creando una forte ridondanza.\\
Inoltre per la maggior parte delle sezioni le relative sotto parti sono visibili grazie all'esposizione tramite elenco puntato, mentre per altre (vedi "Job placement") bisogna cliccare sul titolo della sezione e solo in seguito si viene indirizzati a una pagina contenente le sottosezioni. Questo in fase di scanning potrebbe causare molta confusione: l'utente potrebbe essere indotto a pensare che, per esempio, la sezione "Job placement" non abbia sotto sezioni (anche perchè primo non è scontato che il titolo sia cliccabile, visto che da altre parti con lo stesso pattern non lo è, e secondo perchè vi è già esposto del contenuto, quindi si potrebbe pensare che la struttura della sezione si limiti a riportare il contenuto). \\
In conclusione la strutturazione del contenuto non è per niente intuitiva, probabilmente sono necessari svariati click per permettere all'utente di arrivare all'informazione voluta e la strutturazione dei contenuti nelle pagine interne è ambigua.
