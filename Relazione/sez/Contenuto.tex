Analizziamo in seguito i principali componenti grafici/testuali del sito:
	\begin{itemize}
		\item \textbf{Testo}: in generale per quasi la totalità del testo esiste un adeguato contrasto tra sfondo e colore del testo, questo amplifica la visibilità dello stesso. Non vi è la possibilità di fare la resize del testo, e questo è un fattore negativo; d'altro canto esiste la possibilità di cambiare la lingua in inglese (anche se la traduzione in alcune sezioni lascia un po' a desiderare). Il testo presenta inoltre almeno tre font differenti (parlando di testo della stessa entità) e questo crea confusione e disorientamento. Inoltre, a mio avviso, il testo non è sufficientemente sintetico, anzi, e quindi non permette all'utente di individuare subito i contenuti salienti. Inoltre esso non è diviso in paragrafi e capitoli, e questo influisce negativamente in termini di timer dell'utente;
		\item \textbf{Immagini}: in generale le immagini rispettano la taglia minima consigliata (250x250 px) e sono ben visibili. Sono coerenti con il contenuto del sito (mostrano sedi, studenti che lavorano al computer e laboratori) e nel caso della homepage alleggeriscono un testo fin troppo monolitico. Inoltre esse non sono animate, e questo è un fattore positivo dal momento che il Bloated design è odiato dall'utente;
		\item \textbf{Link}: la gestione dei link (come già accennato in analisi dell'homepage) è totalmente scorretta: essi sono indicati con almeno tre diversi stili e solo uno tra essi rispetti le usuali convezioni per la segnatura dei link, questo non aiuta l'utente a individuarli con facilità. Non è inoltre possibile distinguere i link visitati da quelli no, creando ulteriore confusione. L'unico aspetto positivo è che se clicco uno di essi il suo contenuto non è visualizzato in un'altra scheda (cosa odiata dagli utenti, che amano la navigazione all'indietro) ma in quella corrente.
	\end{itemize}