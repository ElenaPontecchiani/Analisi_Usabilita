Aggiungere immagine con nome del sito?\\
\\ 
Il nome del sito ha un impatto importante (mediamente il 10-20\% in termini di usabilità) e va quindi scelto con cura.\\
Esso deve essere incisivo, facile da ricordare e evocativo rispetto il contenuto che propone. Affinché sia evocativo è bene utilizzare parole di uso comune e che evitino qualsiasi tipo di fraintendimento nell'utente.\\
Nel nostro caso il sito è abbastanza esplicativo rispetto al contesto in cui ci troviamo (dipartimento di Informatica dell'università di Bari), nonostante vi sono alcuni particolari che non lo rendono ottimale:
	\begin{itemize}
		\item \textbf{Presenza di trattini}: in generale non apprezzati all'interno del nome da parte degli utenti;
		\item \textbf{Lunghezza}: il nome è molto lungo, quindi difficile da ricordare;
		\item \textbf{Dominio}: non è \textit{.org}, come è auspicabile che sia;
		\item \textbf{Presenza parole inusuali}: la sigla \textit{Uniba}, che rappresenta l'Università di Bari, potrebbe creare fraintendimenti 												 da parte dell'utente medio, che non necessariamente ne conosce il significato.
	\end{itemize}
In generale, quindi, il nome del sito da un'idea spannometrica del contesto in cui siamo, ma vi è un ampio margine di miglioramento sotto questo punto di vista.