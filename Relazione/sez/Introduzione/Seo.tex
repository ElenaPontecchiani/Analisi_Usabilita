Con \textit{Search Engine Optimization} si intendono tutte quelle attività finalizzate ad ottenere la migliore rilevazione, analisi e lettura del sito web da parte dei motori di ricerca attraverso i loro spider, grazie ad un migliore posizionamento. \\
Il posizionamento di una pagina web è determinato da un "punteggio" ad essa assegnato, detto pagerank.\\
Il pagerank di un sito viene stabilito sulla base due due componenti: la parte testuale e quella ipertestuale.
Al fine di aumentare i punteggi relativi alle parti testuali e ipertestuali esistono diverse tecniche.
Per quanto riguarda la parte testuale si parla della tecnica di \textit{term spam}, che in sostanza consiste nel spammare keyword associate alla pagina web in determinate parti del sito. Ecco dove è preferibile spammare le keyword:
\begin{itemize}
		\item \textsc{Body della pagina}: in generale non bisogna esagerare con questa tecnica, poichè non è molto gradita dall'utente e in 						 generale rischia di intaccare il contenuto della pagina. Nel nostro caso non è praticamente utilizzata questa tecnica;
		
		\item \textsc{Tag meta}: inserendo le keyword nei metatag del sito. Le keyword dovrebbero rispecchiare la maggior parte delle 								 ricerche,costituendo un insieme coeso e di dimensione ridotta. 
					 Nel nostro caso all'interno della pagina HTML del sito non vengono inseriti alcuni tag meta, e questo è un 										grosso punto a sfavore in termini di SEO. Sarebbe stato aspicabile inserire, per esempio, le seguenti keyword: 									\\
						\begin{center}
								Informatica, Uniba, Università, Dipartimento
						\end{center}
						
		\item \textsc{Title spam}: consiste nell'inserire keyword nel tag <title>. Nel nostro caso il tag <title> è
								\\
								\begin{center}
									<title>Dipartimento di Informatica </title>
								\end{center}	
								Il tag quindi è stato utilizzato opportunamente.	
								
		\item \textsc{All'interno di link}: ovvero all'interno dei tag <a> di HTML. Questa tecnica non è utilizzata, ed è un punto a sfavore 						 poichè questa tecnica produce punteggi speciali, in quanto un link è visto come un testo "speciale", più visibile.
		
		\item \textsc{URL spam}: si tratta di inserire le keyword all'interno dell'URL. Attraverso questa tecnica si ottengono bonus simili a 						 quelli del punto precedente. Nel nostro caso nell'url sono inserite keyword adatte, quindi si può dire che la tecnica è 						 stata utilizzata correttamente. 

\end{itemize}

Per quanto riguarda la parte ipertestuale, invece, il punteggio associato è direttamente proporzionale al numero di link entranti e uscenti. Per quanto riguarda i primi non abbiamo gli strumenti per poter arrivare a conclusioni; per quanto riguarda gli outlinks, invece, si nota che sono molto numerosi (in particolare nella homepage), e questo è un aspetto positivo.\\

Alla luce di queste osservazioni analizziamo il posizionamento \textit{SERP} della pagina web.\\
Utilizzando il motore di ricerca google si eseguono delle ricerche di test per valutare la posizione del sito all'interno dei risultati:
	\begin{itemize}
		\item Dipartimento informatica: 16 posto;
		\item Uniba informatica: 1 posto;
		\item Bari informatica: 1 posto;
		\item Aldo Moro informatica: 1 posto (Aldo Moro è il nome dell'università);
		\item Informatica puglia: 19 posto;
		\item Dipartimento informatica Puglia: 2 posto;
	\end{itemize}
Considerando che il 95 per cento dei click dell'utente vanno ai primi 10 risultati, il sito si posiziona bene per ricerche specifiche, mentre con keyword di carattere più generale scende in posizioni molto basse.



		